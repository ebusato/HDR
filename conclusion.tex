\chapter*{Conclusion}
\addcontentsline{toc}{chapter}{Conclusion}

Ce document r\'esume les travaux de recherche que j'ai effectu\'es ces derni\`eres ann\'ees sur l'exp\'erience ATLAS du LHC. 
Ceux-ci ont port\'e sur trois th\'ematiques : la calibration des jets, la recherche de nouvelle physique dans le secteur du quark top et l'interpr\'etation statistique des donn\'ees. 
Pour les deux premi\`eres, les r\'esultats ont \'et\'e obtenus avec, entre autre, deux doctorants que j'ai co-encadr\'es et dont la th\`ese portait en partie sur ces th\'ematiques.
La troisi\`eme th\'ematique n'a pas fait l'objet d'une th\`ese. Sa pr\'esence dans ce manuscript est justifi\'ee par l'importance qu'elle rev\^et dans toute recherche en physique des particules, notamment celle pr\'esent\'ee dans ce manuscript, et par le fait qu'elle a represent\'e une part significative de mon travail ces derni\`eres ann\'ees.

Les travaux r\'ealis\'es sur la calibration des jets ont permis d'\'etablir, avec les premi\`eres donn\'ees du run 1 du LHC, l'int\'er\^et de la m\'ethode de calibration \GS{} (\english{Global Sequential}). 
Nous avons montr\'e notamment que cette calibration est relativement simple \`a mettre en \oe uvre, d\'erivable \`a partir de donn\'ees, performante en terme de r\'esolution en \'energie et entach\'ee d'une faible incertitude syst\'ematique. 
Par ces r\'esultats, nous avons pu poser quelques bases \`a partir desquelles a pu \^etre construite la m\'ethode de calibration actuellement utilis\'ee dans \ATLAS. 
Celle-ci inclut en effet \GS{} dans sa s\'equence de calibration mais au lieu de l'appliquer sur les jets \`a l'\'echelle \'electromagn\'etique, comme nous l'avions fait dans nos \'etudes, elle l'applique aux jets d\'ej\`a calibr\'es par la m\'ethode \LCW. 
Nous pouvons ainsi b\'en\'eficier en m\^eme temps des avantages de la calibration \GS{} et de la calibration des gerbes calorim\'etriques propres \`a la calibration \LCW{}.

Suite aux travaux sur la calibration des jets, mon activit\'e a port\'e sur la recherche de nouvelle physique dans le secteur du quark top. 
Nous avons initi\'e une recherche qui n'avait jamais \'et\'e r\'ealis\'ee jusqu'alors, faute d'une \'energie de collision suffisante : la recherche d'\'ev\'enements avec quatre quarks top dans l'\'etat final. 
Celle-ci a \'et\'e faite dans les \'etats finaux avec deux leptons de m\^eme charge et trois leptons (seuls les \'electrons et muons ont \'et\'e consid\'er\'es). 
L'analyse des donn\'ees d'\ATLAS{} \`a $\sqrt{s}=8~$TeV avec $20,3$~fb$^{-1}$ n'a pas permis d'observer 
%de tels \'ev\'enements. 
de d\'eviation par rapport aux pr\'edictions du mod\`ele standard.
Nous avons par cons\'equent utilis\'e ces donn\'ees pour contraindre des mod\`eles de nouvelle physique pr\'edisant cet \'etat final. 
Le choix des mod\`eles a \'et\'e fait de tel sorte \`a couvrir une grande vari\'et\'e de sc\'enarios de nouvelle physique. 
Pour chacun d'eux, des contraintes am\'eliorant celles existantes ont pu \^etre pos\'ees.
%Plusieurs m\'ethodes statistiques ont \'et\'e utilis\'ees pour \'etablir ces contraintes. 
Les contraintes publi\'ees ont \'et\'e obtenues suivant une approche statistique hybride fr\'equentiste-bay\'esienne. 
Plusieurs \'etudes ont \'et\'e r\'ealis\'ees afin de quantifier leur robustesse.
Elles ont \'et\'e compar\'ees aux contraintes obtenues par des approches fr\'equentiste et bay\'esienne.
Diff\'erents choix pour les distributions \prior{} sur les param\`etres de nuisance ont \'egalement \'et\'e consid\'er\'es.  
Ces \'etudes ont permis de valider les r\'esultats publi\'es \`a une dizaine de pourcent pr\`es
%Les \'ecarts par rapport aux r\'esultats publi\'es pour les sections efficaces exclues n'exc\`edent pas 21\%.

Pour r\'ealiser les calculs de limites d'exclusion, deux outils impl\'ementant les approches hybride et bay\'esienne ont \'et\'e d\'evelopp\'es. 
L'outil hybride, appel\'e \opthylic, a \'et\'e d\'evelopp\'e afin de corriger certains d\'efaults de l'outil hybride de r\'ef\'erence que nous utilisions lorsque nous avons commenc\'e l'analyse (\mclimit). 
\opthylic{} permet des calculs beaucoup plus rapides et plusieurs choix pour les distributions \prior{} sur les param\`etres de nuisance (un seul choix est disponible dans \mclimit). 
L'outil bay\'esien, appel\'e \tifosi, a \'et\'e d\'evelopp\'e afin de pouvoir calculer des limites bay\'esiennes avec le m\^eme mod\`ele statistique que celui utilis\'e dans \opthylic{} (ce mod\`ele \'etant plus pr\'ecis que celui implement\'e dans \histfactory{}, le programme utilis\'e par d\'efault dans \ATLAS). 

Bien que compatible avec les pr\'edictions du mod\`ele standard, le nombre d'\'ev\'enements que nous avons observ\'es dans l'analyse pr\'esente un exc\`es d'environ $2,5\sigma$ par rapport \`a la pr\'ediction de bruit de fond.  %\`a grande \'energie transerve totale, grande \'energie transverse manquante et grand nombre de jets issus de quarks $b$
La recherche d'\'ev\'enements avec quatre quarks top dans le canal avec deux leptons de m\^eme charge et trois leptons va \^etre poursuivie dans \ATLAS{} au run 2 du LHC. 
Nous pourrons ainsi savoir si l'exc\`es observ\'e au run 1 \'etait une fluctuation statistique ou le premier signe d'un ph\'enom\`ene nouveau.
Si les observations restent compatibles avec les pr\'edictions du mod\`ele standard, l'accroissement de l'\'energie de collision et de la luminosit\'e int\'egr\'ee permettra d'am\'eliorer largement les contraintes sur les mod\`eles pr\'edisant l'\'etat final \`a quatre quarks top. 
Certains pourront peut-\^etre \^etre exclus de mani\`ere d\'efinitive.
Si en revanche l'exc\`es persiste et voit sa significance augmenter, alors peut-\^etre que la nouvelle physique tant recherch\'ee commencera \`a r\'ev\'eler certains de ses secrets. 
Et si elle ne le fait pas dans l'\'etat final que nous avons consid\'er\'e, esp\'erons qu'elle le fasse dans un autre.  
Dans tous les cas, avec les nouvelles \'energies atteintes au LHC, nous rentrons dans une phase durant laquelle la compr\'ehension de la physique des particules devrait fortement progresser.


