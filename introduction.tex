\chapter*{Introduction}
\addcontentsline{toc}{chapter}{Introduction}

Ce document r\'esume mes activit\'es de recherche sur l'exp\'erience \ATLAS{} durant le run 1 du LHC. 
Ce run, qui s'est d\'eroul\'e de fin 2009 \`a d\'ebut 2013, correspond \`a la premi\`ere phase de fonctionnement du collisionneur et donc \`a la premi\`ere phase de prise de donn\'ees de l'exp\'erience \ATLAS. 
Il est le premier d'une longue s\'erie qui devrait s'\'ecouler jusqu'en 2030 environ. 
L'\'etude des donn\'ees enregistr\'ees durant le run 1 a permis de r\'ealiser les premi\`eres \'etudes de performance du d\'etecteur apr\`es pr\`es de vingt ann\'ees de R\&D et de construction, 
et a conduit, en 2012, \`a la d\'ecouverte d'un boson scalaire compatible avec le boson BEH. 
Elle a aussi permis  
de r\'ealiser les premi\`eres recherches de ph\'enom\`enes non-pr\'edits par le mod\`ele standard \`a $\sqrt{s}>1.96$~TeV, qui était l'\'energie maximale sond\'ee jusqu'alors gr\^ace au Tevatron et aux exp\'eriences \Dzero et CDF.

Au d\'ebut du run 1, mon activit\'e a port\'e sur la mesure de l'\'energie des jets et plus particuli\`erement sur l'\'etude des performances d'une m\'ethode de calibration permettant %, 
%par l'application de corrections relativement simples, 
une am\'elioration significative de la r\'esolution en \'energie et une r\'eduction importante de l'incertitude syst\'ematique li\'ee \`a la saveur du parton donnant naissance au jet. Ce travail est d\'ecrit dans le chapitre~\ref{chap:calibjets}.

Par la suite mon activit\'e s'est focalis\'ee sur la recherche d'une signature nouvelle de physique au-del\`a du mod\`ele standard : les \'ev\'enements avec quatre quarks top dans l'\'etat final. 
Cette recherche 
%est nouvelle dans le sens o\`u elle 
n'\'etait pas possible au Tevatron du fait de la trop faible \'energie dans le r\'ef\'erentiel du centre de masse. Gr\^ace aux donn\'ees recueillies par l'exp\'erience \ATLAS{} \`a $\sqrt{s}=7$~et~$8$~TeV, 
nous avons 
%ainsi 
pu contraindre des mod\`eles sur lesquels il n'existait pas (ou peu) de contraintes. 
Mon travail a notamment port\'e sur l'interpr\'etation statistique des donn\'ees et plus particuli\`erement sur les m\'ethodes utilis\'ees pour calculer les limites d'exclusion. 
Il a conduit, entre autre, au d\'eveloppement d'outils sp\'ecialis\'es pouvant \^etre utilis\'es pour toute recherche de processus poissonniens. Ces outils sont introduits et d\'ecrits dans les chapitres~\ref{chap:interpretationStatLimit} et \ref{chap:OTHandTIFOSI}. La recherche d'\'ev\'enements avec quatre quarks top ainsi que les r\'esultats obtenus sont d\'ecrits dans le chapitre~\ref{chap:Recherche4tops}.

Les travaux et r\'esultats pr\'esent\'es dans ce document ne sont bien-s\^ur par le fait de mon seul travail. Il s'agit \`a chaque fois de la production d'un groupe de personnes et notamment des doctorants ayant travaill\'es sur ces diff\'erents sujets. Ces personnes (au moins celles avec qui j'ai le plus travaill\'e) seront cit\'ees d\`es qu'il se doit au sein de ce document.

